\documentclass{article}
\usepackage[utf8]{inputenc}
\usepackage[spanish]{babel}
\usepackage{listings}
\usepackage{graphicx}
\graphicspath{ {images/} }
\usepackage{cite}

\begin{document}

\begin{titlepage}
    \begin{center}
        \vspace*{1cm}
            
        \Huge
        \textbf{Taller - Nociones de la memoria }
            
        \vspace{0.5cm}
      
            
        \vspace{1.5cm}
            
        \textbf{Deybison Camilo Sepúlveda Vélez}\\
             \LARGE
        C.C. 1039597966
            
        \vfill
            
        \vspace{0.8cm}
            
        \Large
        Despartamento de Ingeniería Electrónica y Telecomunicaciones\\
        Universidad de Antioquia\\
        Medellín\\
        Septiembre de 2020
            
    \end{center}
\end{titlepage}

\tableofcontents

\section{Defina que es la memoria del computador.}
La palabra memoria se utiliza para referirse a las partes de un computador en donde se almacenan todo tipo de datos, archivos y programas. Este hardware es uno de los principales componentes de un computador puesto que, sin la memoria la computadora no funcionaría dado que este componente es la que almacena y ejecuta las instrucciones del computador o CPU disponibles. La memoria se encuentra dentro del gabinete, acoplada a la placa madre.


\section{Mencione los tipos de memoria que conoce y haga una pequeña descripción de cada tipo.}\\
\textbf{RAM:} Random Access Memory) en esta memoria se alojan procesos temporales, en esta memoria se puede leer y escribir todo tipo de datos, desde modificaciones en los archivos y también todas las instrucciones que se necesitan para la ejecución de aplicaciones.\\
\textbf{ROM:} (Read Only Memory) esta memoria almacena todos los procesos de arranque de un dispositivo electrónico, sus datos almacenados se mantienen aun cuando no tienen una fuente de energía.\\
\textbf{Memoria Cache:} en la memoria cache se alojan temporalmente todos los datos que fueron procesados recientemente, permite que el microprocesador acceda a los datos que se usan más frecuentemente.\\
\textbf{Memoria flash:} este tipo de memoria funciona como dispositivo de transporte y almacenamiento de información.\\
\textbf{Memoria Virtual:} es una porcion de discoduro inplementado para guardar temporalmente porciones de programas y datos en ejecucion que se utilizan menos.


\section{Describa la manera como se gestiona la memoria en un computador.}
la gestion de la memoria empieza con una orden de un usuario, esta orden viaja a traves de un periferico de entrada guardando esta orden temporalente en un espacio de la memoria, despues al microprocesador se le da aviso de que en algun espacio en la memoria hay una orden que necesita ser ejecutada el microprocesador la busca, esta orden es eliminada del microprocesador y de la memoria y es llevado por un controlador a unespacio de memoria en el cual se puede seguir ejecutando.
\section{¿Qué hace que una memoria sea más rápida que otra? ¿Por qué esto es importante?}
Lo que hace que una memoria sea mas rapida que otra son dos cosas la velocidad del bus de la placa madre, mas precisamente de la velocidad del reloj del bus de la placa madre y de la cantidad de bits que se transfieren a la vez a través del bus.
Esto es importante ya que entre más veloz sea la memoria más rápido será el computador para almacenar, recuperar y transferir datos, y al computador le sera mas facil mantener y ejcutar diferentes programas a la vez.\\
\\
\\

\bibliographystyle{IEEEtran}
\bibliography{references}
\begin{itemize}
\item https://tarjetasgraficaspc.com/memorias-ram/latencia-velocidad-memoria-ram/
\item https://compuline.com.mx/blog/memoria-ram-y-su-velocidad/
\item https://sistemasoperativosfesaragon.wordpress.com/unidad-5-gestion-de-la-memoria/
\item https://www.tecnologia-informatica.com/tipos-memorias-computadora/#:~:text=No%20obstante%2C%20una%20computadora%20trabaja,memoria%20Virtual%20o%20de%20Swap.
\item https://www.ecured.cu/Memoria_(inform%C3%A1tica)
\end{itemize}
\end{document}
